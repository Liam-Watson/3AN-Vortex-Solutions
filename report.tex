\documentclass{article}
\DeclareMathSizes{10}{10}{7}{7}
\usepackage{amsmath}
\usepackage{ amssymb }
\usepackage{tikz, graphicx}
\usepackage{geometry}
\usepackage[makeroom]{cancel}
% http://www.mathworks.com/matlabcentral/fileexchange/8015-m-code-latex-package
\usepackage[framed,numbered,autolinebreaks,useliterate]{mcode}
\geometry{legalpaper, margin=0.7in}
\title{3AN Project 1\\ Shooting Newton and Kantorovich with matlab}
\author{Liam Watson WTSLIA001}
\begin{document}
\maketitle
When modeling vortecies a common Boundary Value problem that results is as follows


\begin{align}
&\frac{d^2 u}{dr^2} + \frac{1}{r}\frac{du}{dr} + \frac{u}{1-u^2}\left[\left(\frac{du}{dr}\right)^2 - \frac{n^2}{r^2}\right] + u(1-u^2) = 0 \\
&u(0) = 0,\ \ \  u(\infty) = 1
\end{align}

Where 
\begin{align*}
&n\in \mathbb{N}/\{0\} \text{ Is the vorticity} \\
&u = u(r)  \text{ Is our unknown function that is monotically growing} 
\end{align*}
In this paper we will solve the BVP (1), (2) numerically using the shooting method and the Newton-Kantorovich method for $n\in \{1,2,3\}$. The numerical solution to this BVP is sensitive to many parameters that we must choose. We will discuss the effect of our choices for parameters such as how we deal with $u(\infty)$, $\frac{u1}{1-u^2}$ being undefined at $u = 1$, $u(0) = 0$ but $1/r$ is undefined, the number of itterations, the value for $n$, mesh spacing, length of spacial interval, initial guess.

\section{Numerical methods overview}
Here we will discuss the different numerical schemes with their relative benefits and drawbacks 
\subsection{Shooting method}
The general principle for the shooting method is to reduce our boundary value problem into an initial value problem. We do this by taking the first boundary condition as the first initial condition and some arbitrary second value for the second initial condition, we call this value the shooting parameter. We must then find a value of the shooting parameter so that our second boundary condition is met. We can do thing using the bisection or netwton's method.     
\subsection{Newton-Kantorovich method}
The general principle for Newton-Kantorovich is to find the system of Kantorovich equations. In our case this will produce a second order linear boundary value problem which we can solve using linear shooting or finite difference methods. The solution to this system is the correction to some initial guess for the initial problem to be solved.  
\section{Derevation of needed formula}
In order to begin the implementation we need to first derive the needed expressions from (1) and (2).
\section{Shooting method}
The differential equation in (1) is a non-linear one so we must use non-linear shooting to solve it. \\
First we must derive an initial value problem from the boundary value problem which is described bellow 
\begin{align}
\begin{cases}
u'' = -\frac{1}{r}\frac{du}{dr} - \frac{u}{1-u^2}\left[\left(\frac{du}{dr}\right)^2 - \frac{n^2}{r^2}\right] - u(1-u^2) \\
u(0) = 0 \\
u'(0) = p
\end{cases}
\end{align}
Where $p$ is our shooting parameter. \\
We now have two choice for solving for $p$ namely, bisection or newton's method. 
We do not yet know which of these will give better results so let us implement both.  \\
For bisection there is nothing for us do derive and the details will be covered in the implementation section. \\
For Newtons method we must derive a scheme to update $p$ based on the function $u$ and the seccond boundary condition.\\ 
We introduce:
\begin{align}
f' = \frac{\partial }{\partial p}\left(u(p,x=\infty)-1\right) =  \frac{\partial }{\partial p} u(p,x=\infty) \\
z(p) = \frac{\partial }{\partial p} u(p,x)
\end{align}
Then 
\begin{align}
\begin{cases}
z'' = \frac{\partial f}{\partial y}z + \frac{\partial f}{\partial y'}z' \\
 \frac{\partial f}{\partial y} = \\
 \frac{\partial f}{\partial y'} = \\
\end{cases}
\end{align}
We can use reduction of order to obtain the systems we must solve numerically. 
\begin{align}
\begin{cases}
u' = v \\
v' = -\frac{1}{r}\frac{du}{dr} - \frac{u}{1-u^2}\left[\left(\frac{du}{dr}\right)^2 - \frac{n^2}{r^2}\right] - u(1-u^2) \\
z' = w \\
w = \left(\frac{1+u^2}{(1-u^2)^2}\left[ v^2 + \frac{n^2}{r^2} \right] \right)z + \left(\frac{-1}{r} + \frac{2uv}{1-u^2}\right)w \\
\end{cases}
\end{align}
\section{Newton-Kantorovich}
The first step for the Newton-Kantorovich is to find the Kantorovich equations which we can do by:
\begin{itemize}
\item Find $F(u) = 0$
\item Find $F(y + hz)$
\item Find $\frac{\partial F}{\partial h}$
\item Find $\lim_{h\rightarrow 0} \frac{\partial F}{\partial h}$
\end{itemize}
\begin{enumerate}
\item
\begin{align}
F(y) = 
\begin{cases}
\frac{d^2 u}{dr^2} + \frac{1}{r}\frac{du}{dr} + \frac{u}{1-u^2}\left[\left(\frac{du}{dr}\right)^2 - \frac{n^2}{r^2}\right] + u(1-u^2) = 0 \\
u(0) = 0 \\
u(\infty) = 1\\
\end{cases}
\end{align}
\item
\begin{align}
F(y+hz) = 
\begin{cases}
u\prime\prime + hz\prime\prime + \frac{1}{r}\left(u\prime + hz\prime\right) + \frac{u + zh}{1-(u+zh)^2}\left[(u\prime + hz\prime)^2 - \frac{n^2}{r^2} \right] + (u + hz)(1-(u+hz)^2) = 0 \\
u(0) + hz(0) = 0 \\
u(\infty) + hz(\infty) = 1\\
\end{cases}
\end{align}
\item and 4. assuming we know some approximation $u^n$
\begin{align}
\begin{cases}
z\prime\prime + \frac{1}{r}z\prime + z(1-3u^2) + \frac{1}{1-u^2}\left[ (u\prime)^2  - \frac{n^2}{r^2}\right] = -(u^n)\prime\prime - \frac{1}{r}(u^n)\prime - \frac{u^n}{1-(u^n)^2}\left[\left((u^n)^\prime\right)^2 - \frac{n^2}{r^2}\right] - u^n(1-(u^n)^2) \\
z(0) = -u^n(0) \\
z(\infty) = u^n(\infty) + 1 
\end{cases}
\end{align}
This system (10) is now linear in $z$ so we can solve it using any numerical scheme for linear BVP's. Knowing that the standard form for a linear BVP is
\begin{align}
u^{\prime\prime} = p(r)u^\prime + q(r)u + \xi (r)
\end{align}
Trivially we can find the corresponding $p,q,\xi$ for (10)
\begin{align*}
\begin{cases}
p = \frac{1}{r} _ \frac{2uu^\prime}{1-u^2} \\
q = -3u^2 + 1 + \frac{1}{1-u^2}\left[(u^\prime)^2 -\frac{n^2}{r^2} \right] \\
\xi = u
\end{cases}
\end{align*}
We choose to solve for $z$ using the finite difference method. First we must define a mesh.
\begin{align}
h = \frac{1-0}{N} \ \ \ \ \ \ \text{Where we are free to choose a large N} \\
x_j = (j-1)*h, \ \ \ j = 1,2,3,...,N+1 \\
u_j = u(x_j)
\end{align}
 
We use finite difference approxmiations for $u^\prime$ and $u^{\prime\prime}$
\begin{align}
(u_j)^\prime = \frac{u_{j+1} - u_{j-1}}{2h} \\
(u_j)^{\prime\prime} = \frac{u_{j+1} - 2u_{j} + u_{j-1}}{h^2} \\
\end{align}
We then have our matrix-vector system defined as
\begin{align}
\begin{bmatrix}
1 & 0 & 0 ... 0 \\
1 + \frac{hp_3}{2} & -2 - h^2q_3 & 1 - \frac{hp_3}{2} & 0 &...& 0\\
0 & . & . \\
. & . & . \\
. & . & . \\
0 & ... &0 & & 1 + \frac{hp_N}{2} & -2 - h^2q_3 \\
\end{bmatrix}
\begin{bmatrix}
z_2 \\ z_3 \\ . \\ . \\ . \\ z_N\\
\end{bmatrix}
=
\begin{bmatrix}
h^2 \xi \\
h^2 \xi \\
. \\
. \\
. \\
h^2 \xi - 1
\end{bmatrix}
\end{align}
Which we can solve for $z$ and simplify in matrix-vector notation to
\begin{align}
\vec{z} = \vec{U} A^{-1} 
\end{align}
We then wish to update $u$ using the correction $z$
\begin{align}
u^{n + 1} = u^{n} + z
\end{align}
\end{enumerate}
\section{Implementation}
Here we will discuss the implementation details of the two numerical methods in matlab. \\

\subsection{Shooting method}
\begin{lstlisting}
a = 0.1; % Boundary one
b = 8; % infinity
n = 1;
%p = 0.2; %Arbitray initial shooting parameter


%file = fopen("inf.txt", 'w');
%fprintf(file,'%6.2f %12.8f\n', n);
%for b = [5:1:25]
    ph = 2; %bisection
    pl = 0;
    p = (ph + pl)/2; %bisection
    tol = 1;
    i = 1;
    u0 = [0,p,0,1];
while i < 300 && tol > 1e-5
   [r,u] = ode45(@shoot, [a:0.2:b], u0);
   %disp(u(end,3))
   plot(r,u(:,1));
   %pause(0.1)
   %disp((u(end,4)))
   tol = abs(1-u(end,1));
   tor = u(end,1);
   if(tor > 1)
      ph = p;
      p = (pl + ph)/2;
      u0 = [0,p,0,1];   
   end
   if(tor < 1)
      pl = p;
      p = (pl + ph)/2;
      u0 = [0,p,0,1];  
   end
   %disp(p)
   %p = p - ((u(end,1)-1)/(u(end,3)));
   %u0 = [0,p,0,1];
   %disp([p,ph,pl])
   %disp(u0);
   i = i + 1;
   plot(r,u(:,1))
   
end
%    bool = tol < 1e-5;%tolerence to be considered stable
%    fprintf(file,'%6.2f %12.8f\n', b, bool);
%end
plot(r,u(:,1))
%fclose(file);
function [du] = shoot(r,u)
    n = 1;
    uu = u(1);
    v = u(2);
    z = u(3);
    w = u(4);
    
    
    du(1,1) = v; %
    du(2,1) = (-1/r)*v  - (uu/(1-uu^2))*(v^2-(n^2)/(r^2)) - uu*(1-uu^2);
    du(3,1) = w; % 
    du(4,1) = z*( -((1 + uu^2)/((1-uu^2)^2))*(v^2 - (n^2)/(r^2)) - 1 + 3*uu^2) + w*(-1/r  - ((2*uu)*v)/(1-uu^2)) ;
    
end
\end{lstlisting}
\pagebreak
\subsection{Newton-Kantorovich method}
\begin{lstlisting}

%h = (b-a)/N;

%r = (a:h:b)';

%Initial guesses
%u = r/b;
%u = log(r + 1)/log(b+1) -a; 
%u = 1./(1+exp(-r + b/2)); 

n = 1; %vorticity 
%file = fopen("inf.txt", 'w');
%for N = [50:100:1950] %Here we loop over our hyperparameters and test for
%stability i.e. N, b, a, etc.
    a = 0.1; %This is our starting position
    b = 5; %infity
    N = 100;
    h = (b-a)/N;
    h2 = h*h;
    r = (a:h:b)';
    u = r/b;
    %u = 1./(1+exp(-r));
    rhs=[u(1);u(1:N-1)-2*u(2:N)+u(3:N+1) + h2.*f(r,u,h, N, n);u(N+1)-1];
    tol = 1;
    count = 1;

while count < 500 && tol > 1e-6
    dd = [0; -2 - h2.*q(r,u,h,N, n) ; 0];
    upper = diag([0;1 - (h*p(r,u,h,N))./2],1);
    lower = diag([1 + (h*p(r,u,h,N))./2;0],-1);
    J = diag(dd,0) + upper + lower;  %Set three diagonal
    J(1,1)=1; J(1,2)=0; J(1,3)=0; %Set BC 1
    J(N+1,N+1)=1; J(N+1,N)=0; J(N+1,N-1)=0; %Set BC 2
    z = -J\rhs; %Solve for correction
    u = u + z; 
    rhs=[u(1);u(1:N-1)-2*u(2:N)+u(3:N+1) + h2.*f(r,u,h,N,n);u(N+1)-1]; 
    tol = norm(rhs) %Display tolerance
    plot(r,u)
    pause(0.05);
    count = count+1;
    
end
%    bool = tol < 1e-5;%tolerence to be considered stable
%    fprintf(file,'%6.2f %12.8f\n', b, bool); %Write stability data
%end %End of for loop
%fclose(file);
%This is f in y'' = f
function [fu] = f(r, u, h, N, n)
    uBefore = u(1:N-1);
    uAfter = u(3:N+1);
    u = u(2:N);
    fu = 1*((1./r(2:N)).*((uAfter - uBefore)/(2*h)) + ((u)./(1-u.^2)).*(((uAfter - uBefore)/(2*h)).^2 - n^2./(r(2:N).^2)) + u.*(1-u.^2));
end 
%This is p in a linear ODE
function [p] = p(r,u,h, N)
    uBefore = u(1:N-1);
    uAfter = u(3:N+1);
    uu = u(2:N);
    r = r(2:N);
    p = -1*(1./r + (2*uu.*((uAfter-uBefore)/(2*h)))./(1-uu.^2));
end


%q(r)
function [q] =  q(r,u, h , N, n)
    r = r(2:N);
    uBefore = u(1:N-1);
    uAfter = u(3:N+1);
    uu = u(2:N);
    q= -1*(-3*uu.^2 + 1 + 1./((1-uu.^2)).*(((uAfter-uBefore)/(2*h)).^2 - (n^2)./(r.^2) )); 
end
\end{lstlisting}
\pagebreak
\section{Results and discussion}
Here we will discuss how the two methods performed and what we learned. \\
The testing stratergy for the hyperparameters was as follows: We must have base stable parameters that we can fix while we vary the others. Let these be

\begin{align*}
 & a = 0.1 \\
 & b = 5 \\
 & N = 50 \\
 & step size = 0.2 \\
 & u^0 = \frac{r}{b} \\
 & p0 = 0.2 \text{ \ \ \ \ For Newton's method in Nonlinear   shooting}\\
 & ph = 4 \text{ \ \ \ \ For bisection in Nonlinear shooting}\\
 & pl = 4 \text{ \ \ \ \ For bisection in Nonlinear shooting}
\end{align*}
Each test will be run until either 100 iterations is reached or the tolerance less than $10^{-7}$. \\
The parameters that we will analyse for stability are $a,b,N,$ step size$,u^0$ and bisection vs newton.
\subsection{Shooting method}
\subsubsection{Varing a}
\begin{table}[]
\begin{tabular}{ll}
1     &   \\
0     & 0 \\
0.001 & 1 \\
0.002 & 1 \\
0.003 & 1 \\
0.004 & 1 \\
0.005 & 1 \\
0.006 & 1 \\
0.007 & 1 \\
0.008 & 1 \\
0.009 & 1 \\
0.01  & 1 \\
      &   \\
      &   \\
      &   \\
2     &   \\
0     & 0 \\
0.001 & 1 \\
0.002 & 1 \\
0.003 & 1 \\
0.004 & 1 \\
0.005 & 1 \\
0.006 & 1 \\
0.007 & 1 \\
0.008 & 1 \\
0.009 & 1 \\
0.01  & 1 \\
      &   \\
3     &   \\
0     & 0 \\
0.001 & 0 \\
0.002 & 0 \\
0.003 & 0 \\
0.004 & 0 \\
0.005 & 0 \\
0.006 & 0 \\
0.007 & 0 \\
0.008 & 0 \\
0.009 & 1 \\
0.01  & 1
\end{tabular}
\end{table}
\subsubsection{Varing b}
\begin{table}[]
\begin{tabular}{ll}
1  &   \\
4  & 1 \\
6  & 1 \\
8  & 1 \\
10 & 1 \\
12 & 1 \\
14 & 1 \\
16 & 0 \\
18 & 0 \\
20 & 0 \\
22 & 0 \\
24 & 0 \\
   &   \\
2  &   \\
4  & 1 \\
6  & 1 \\
8  & 1 \\
10 & 1 \\
12 & 1 \\
14 & 1 \\
16 & 1 \\
18 & 0 \\
20 & 0 \\
22 & 0 \\
24 & 0 \\
   &   \\
3  &   \\
4  & 1 \\
6  & 1 \\
8  & 1 \\
10 & 1 \\
12 & 1 \\
14 & 1 \\
16 & 1 \\
18 & 0 \\
20 & 0 \\
22 & 0 \\
24 & 0
\end{tabular}
\end{table}
\subsubsection{Varing step size}
Current step size testing did not yeild any interesting results. 
\subsubsection{Varing bisection vs Newton}
TBD.
\includegraphics[scale=1]{nonlinearshoot1.png}
\subsection{Newton-Kantorovich method}
\subsubsection{Varing a}
\begin{table}[]
\begin{tabular}{ll}
1  &   \\
4  & 1 \\
6  & 1 \\
8  & 1 \\
10 & 1 \\
12 & 1 \\
14 & 1 \\
16 & 0 \\
18 & 0 \\
20 & 0 \\
22 & 0 \\
24 & 0 \\
   &   \\
2  &   \\
4  & 1 \\
6  & 1 \\
8  & 1 \\
10 & 1 \\
12 & 1 \\
14 & 1 \\
16 & 1 \\
18 & 0 \\
20 & 0 \\
22 & 0 \\
24 & 0 \\
   &   \\
3  &   \\
4  & 1 \\
6  & 1 \\
8  & 1 \\
10 & 1 \\
12 & 1 \\
14 & 1 \\
16 & 1 \\
18 & 0 \\
20 & 0 \\
22 & 0 \\
24 & 0
\end{tabular}
\end{table}
\subsubsection{Varing b}
\begin{table}[]
\begin{tabular}{ll}
1  &   \\
4  & 1 \\
6  & 0 \\
8  & 0 \\
10 & 0 \\
12 & 0 \\
14 & 0 \\
16 & 0 \\
18 & 0 \\
20 & 0 \\
22 & 0 \\
24 & 0 \\
2  &   \\
4  & 1 \\
6  & 1 \\
8  & 0 \\
10 & 0 \\
12 & 0 \\
14 & 0 \\
16 & 0 \\
18 & 0 \\
20 & 0 \\
22 & 0 \\
24 & 0 \\
3  &   \\
4  & 1 \\
6  & 1 \\
8  & 1 \\
10 & 1 \\
12 & 0 \\
14 & 0 \\
16 & 0 \\
18 & 0 \\
20 & 0 \\
22 & 0 \\
24 & 0
\end{tabular}
\end{table}
\subsubsection{Varing N}
\begin{table}[]
\begin{tabular}{ll}
1    &   \\
10   & 0 \\
110  & 0 \\
210  & 0 \\
310  & 0 \\
410  & 0 \\
510  & 0 \\
610  & 0 \\
710  & 0 \\
810  & 0 \\
910  & 1 \\
1010 & 0 \\
2    &   \\
10   & 1 \\
110  & 1 \\
210  & 1 \\
310  & 1 \\
410  & 1 \\
510  & 1 \\
610  & 1 \\
710  & 1 \\
810  & 1 \\
910  & 1 \\
1010 & 1 \\
3    &   \\
10   & 1 \\
110  & 1 \\
210  & 1 \\
310  & 1 \\
410  & 1 \\
510  & 1 \\
610  & 1 \\
710  & 1 \\
810  & 1 \\
910  & 1 \\
1010 & 1
\end{tabular}
\end{table}
\subsubsection{Varing initial guess}
\begin{table}[]
\begin{tabular}{ll}
                                           &   \\
guess 1=linear, 2=sigmoid, 3=-e-r+1, 4=log &   \\
1                                          &   \\
1                                          &   \\
4                                          & 1 \\
6                                          & 0 \\
8                                          & 0 \\
10                                         & 0 \\
12                                         & 0 \\
14                                         & 0 \\
16                                         & 0 \\
2                                          &   \\
4                                          & 1 \\
6                                          & 0 \\
8                                          & 0 \\
10                                         & 0 \\
12                                         & 0 \\
14                                         & 0 \\
16                                         & 0 \\
3                                          &   \\
4                                          & 1 \\
6                                          & 0 \\
8                                          & 0 \\
10                                         & 0 \\
12                                         & 0 \\
14                                         & 0 \\
16                                         & 0 \\
4                                          &   \\
4                                          & 1 \\
6                                          & 0 \\
8                                          & 0 \\
10                                         & 0 \\
12                                         & 0 \\
14                                         & 0 \\
16                                         & 0 \\
2                                          &   \\
1                                          &   \\
4                                          & 1 \\
6                                          & 1 \\
8                                          & 0 \\
10                                         & 0 \\
12                                         & 0 \\
14                                         & 0 \\
16                                         & 0 \\
2                                          &   \\
4                                          & 1 \\
6                                          & 1 \\
8                                          & 0 \\
10                                         & 0 \\
12                                         & 0 \\
14                                         & 0 \\
16                                         & 0 \\
3                                          &   \\
4                                          & 1 \\
6                                          & 1 \\
8                                          & 1 \\
10                                         & 0 \\
12                                         & 0 \\
14                                         & 0 \\
16                                         & 0 \\
4                                          &   \\
4                                          & 1 \\
6                                          & 0 \\
8                                          & 0 \\
10                                         & 0 \\
12                                         & 0 \\
14                                         & 0 \\
16                                         & 0 \\
3                                          &   \\
1                                          &   \\
4                                          & 1 \\
6                                          & 1 \\
8                                          & 1 \\
10                                         & 1 \\
12                                         & 0 \\
14                                         & 0 \\
16                                         & 0 \\
2                                          &   \\
4                                          & 1 \\
6                                          & 1 \\
8                                          & 1 \\
10                                         & 0 \\
12                                         & 0 \\
14                                         & 0 \\
16                                         & 0 \\
3                                          &   \\
4                                          & 1 \\
6                                          & 1 \\
8                                          & 1 \\
10                                         & 1 \\
12                                         & 0 \\
14                                         & 0 \\
16                                         & 0 \\
4                                          &   \\
4                                          & 1 \\
6                                          & 0 \\
8                                          & 0 \\
10                                         & 0 \\
12                                         & 0 \\
14                                         & 0 \\
16                                         & 0
\end{tabular}
\end{table}
\includegraphics[scale=1]{NK.png}
\section{Conclusion and final remarks}
GGGGGGG
\end{document}